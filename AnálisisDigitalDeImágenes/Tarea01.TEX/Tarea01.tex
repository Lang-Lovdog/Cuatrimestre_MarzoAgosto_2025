%! TeX program = pdflatex
%! TeX TS-program = pdflatex

\documentclass[10pt]{IEEEtran}
\usepackage{amsmath}
\usepackage[utf8]{inputenc}
\usepackage{fontenc}
\usepackage{dirtytalk}
\usepackage[dvipsnames]{xcolor}
\usepackage[cmintegrals]{newtxmath}
\usepackage{tikz}
\usepackage{varwidth}
\usepackage{graphicx}
\usetikzlibrary{positioning}
\tikzset{
  x=1em,
  y=1em,
}

\title{Encontrando el \say{esqueleto} de un dibujo}

\author{ Brandon Marquez Salazar }

\begin{document}
  \maketitle
  \section{Introducción}
  En este documento se muestra el procedimiento en el que se pueden encontrar los trazos básicos de un dibujo o su "esqueleto". Este es un problema típico que se puede encontrar en el área de artes digitales
  el momento de querer digitalizar, esbozar o mejorar una imagen.

  \section{Procedimiento}
  Para realizar este proceso se propuso el uso de un kernel de 3x3 de la forma
  \begin{tabular}{ c c c }
    -2 & -2 &  0\\
    -2 &  0 & -2\\
     0 & -2 & -2
  \end{tabular}
  
\end{document}

