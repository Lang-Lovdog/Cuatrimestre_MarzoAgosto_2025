%! TeX program = pdflatex
%! TeX TS-program = pdflatex

\documentclass[10pt]{IEEEtran}
\usepackage{amsmath}
\usepackage[utf8]{inputenc}
\usepackage{fontenc}
\usepackage{dirtytalk}
\usepackage[dvipsnames]{xcolor}
\usepackage{tikz}
\usepackage{varwidth}
\usepackage{csquotes}
\usepackage{listings}
\usepackage{graphicx}
\usepackage[spanish]{babel}
\usetikzlibrary{positioning}
\usepackage{biblatex}
\addbibresource{bibl.bib}
%% Add latin special characters for lstlisting
\lstset{literate={á}{{\'a}}1 {é}{{\'e}}1 {ñ}{{\~n}}1}
\tikzset{
  x=1em,
  y=1em,
}
\lstset{
  basicstyle=\ttfamily\footnotesize,
  showstringspaces=false,
  commentstyle=\itshape,
  keywordstyle=\bfseries\color{cyan},
  stringstyle=\ttfamily,
  language=C,
  tabsize=2,
  breaklines=true,
  breakatwhitespace=true,
  escapeinside={\%*}{*)},
  morekeywords={
    cv,
    filter2D,
    abs,
    NORM_MINMAX,
    CV_8UC1,
    THRESH_BINARY,
    threshold,
    Point,
    BORDER_DEFAULT,
    normalize,
    equalizeHist
  },
}

\graphicspath{{code/res/}}

\title{Clasificación utilizando momentos de Hu}

\author{ Brandon Marquez Salazar }

\begin{document}
  \maketitle
  \section{Introducción}

  \section{Materiales y métodos}
  El programa se realizará utilizando el lenguaje C++ y la biblioteca OpenCV para la manipulación de gráficos.
  El algoritmo de transferencia de color utilizado fue desarrollado en \cite{Reinhard2001}.
  Se utilizarán imágees umbralizadas con formas que serán clasificadas según una tabla de características dada.

  \section{Objetivos}
  El objetivo es la correcta clasificación de las imágenes t1 y t2 utilizandos los momentos de invariantes de Hu,
  y comparándolos con los valores dentro de una tabla de características.
  
  \begin{table}[htpb]
    \centering
    \caption{Prototipos de clase para seis formas}
    \label{tab:prototipos}
  
    \begin{tabular}{ c l l l l l l l }
      id &    H0  &   H1   &   H2   &   H3   &   H4    &   H5   &   H6    \\
      1: &  0.49  &  2.00  &  1.96  & 3.79   & 6.82    &  4.95  & -6.80   \\
      2: &  0.22  &  0.49  &  2.51  &  3.05  &  5.92   &  3.59  & -6.05   \\
      3: &  0.55  &  3.08  &  4.90  &  5.35  & -10.64  &  6.90  & -10.61  \\
      4: &  0.44  &  1.55  &  2.52  &  3.76  &  7.21   &  4.96  &  6.95   \\
      5: &  0.42  &  1.40  &  2.72  &  4.13  &  7.55   &  4.84  &  9.07   \\ 
      6: &  0.46  &  1.74  &  1.89  &  2.59  & -4.88   & -3.49  & -5.15   \\
    \end{tabular}
  \end{table}

  \section{Procedimiento}
  Sea $c$ una clase de la tabla de prototipos e $i$ una imagen a clasificar, entonces tenemos que la similitud
  $S_{c,i}$ se calcula obteniendo la distancia entre el momento $m$ de la imagen $i$ y el prototipo $c$.
  Cabe indicar que para una $S_{c,i}$ menor, se considera una mayor similitud.

  \begin{equation}
  \begin{aligned}
    S_{c,i} = \sum_{m=0}^{6} {|H_{c,m} - H_{i,m}|}
  \end{aligned}
  \end{equation}

  \includegraphics[width=0.4\linewidth]{t1.jpg}
  \includegraphics[width=0.4\linewidth]{t2.jpg}

  \section{Resultados}
  Se obtuvo que las imágenes t1 y t2 son de las clase 6.
  %% Write output as format of code
  \begin{lstlisting}[language=C]
Imagen t1
Similitud listada a la clase del contorno 1 (valor más pequeño, similitud mayor):
S[1] = 36.5689
S[2] = 33.5491
S[3] = 38.2513
S[4] = 24.7376
S[5] = 27.6618
S[6] = 14.5171
Clase más similiar: 6

Imagen t2
Similitud listada a la clase del contorno 1 (valor más pequeño, similitud mayor):
S[1] = 52.0064
S[2] = 50.9664
S[3] = 37.4832
S[4] = 38.6264
S[5] = 36.3264
S[6] = 31.7764
  \end{lstlisting}

  \section{Conclusiones}
  Los momentos de Hu parecen haber clasificado de la misma forma dos diferentes
  imágenes, es importante resaltar la necesidad de comprender que los prototipos
  mostrados no tienen mayor contexto que la tabla, por lo que únicamente podemos
  quedarnos con los valores numéricos.

  También debe considerarse un error al tratar los momentos de Hu al definir la
  ecuación de error o al implementarla.

  \printbibliography
\end{document}
