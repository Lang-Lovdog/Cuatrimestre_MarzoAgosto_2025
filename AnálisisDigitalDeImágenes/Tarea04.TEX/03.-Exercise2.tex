%!TeX root = MarquezSalazarBrandon-Tarea04-Reporte.tex
\section{ Implementation of the SDH algorithm}


For this experiment, it was implemented a class \textbf{SDH},
which uses some of the \textbf{OpenCV} library functions.
The purpose is to compute texture properties from an image,
for specific $d$ and $\theta$ 

First, of all, the library includes to assure all it needs to
work properly.

\begin{file}[SDH\_feat.hxx]
  \begin{lstlisting}[language=C]
#ifndef __LOVDOG_SDH_FEAT_HXX__
#define __LOVDOG_SDH_FEAT_HXX__
#include <opencv2/highgui.hpp>
#include <opencv2/imgproc.hpp>
#include <opencv2/core.hpp>
#include <vector>
#include <string>
  \end{lstlisting}
\end{file}

Now, the class \textbf{SDH} was declared inside a namespace
(\textbf{lovdog}) with the following attributes:

\subsection{Public attributes}
As a class, it's needed to define some public attributes and
methods to be accesible to the user.

\subsubsection{Static constants}
Static constants for angle and histogram index.
\begin{file}[SDH\_feat.hxx]
  \begin{lstlisting}[language=C]
      static const uint
        ANGLE_0=0,
        ANGLE_45=45,
        ANGLE_90=90,
        ANGLE_135=135,
        ANGLE_180=180
      ;
      static constexpr uint ANGLE[5] = {
        ANGLE_0,
        ANGLE_45,
        ANGLE_90,
        ANGLE_135,
        ANGLE_180
      };
      static const int
        DIFF=0,
        SUM=1
      ;
  \end{lstlisting}
\end{file}

\subsubsection{Control variables and macros} \label{control_variables}
Within the class, is also available some header macros for
visualization and exports. And two control variables for log and
verbosity.
\begin{file}[SDH\_feat.hxx]
  \begin{lstlisting}[language=C]
      bool logOn;
      char verbose;
      static const uint
        ALL            = ~0,
        _MEAN_DIFF     = 0b000000000001,
        _MEAN_SUM      = 0b000000000010,
        _VARIANCE_DIFF = 0b000000000100,
        _VARIANCE_SUM  = 0b000000001000,
        CORRELATION    = 0b000000010000,
        CONTRAST       = 0b000000100000,
        HOMOGENEITY    = 0b000001000000,
        SHADOWNESS     = 0b000010000000,
        PROMINENCE     = 0b000100000000,
        ENERGY         = 0b001000000000,
        ENTROPY        = 0b010000000000,
        MEAN           = 0b100000000000
      ;
  \end{lstlisting}
\end{file}

Some attributes for features and histograms.
\begin{file}[SDH\_feat.hxx]
  \begin{lstlisting}[language=C]
      cv::Mat src;
      uint
        d,
        angle
      ;
      int
        dx,
        dy
      ;
      double
        _mean[2],
        _variance[2],
        correlation,
        contrast,
        homogeneity,
        shadowness,
        prominence,
        energy,
        entropy,
        mean
      ;
  \end{lstlisting}
\end{file}
\subsection{Private attributes}
There are few private attributes for the class,
which are the histograms array and two pointers to each
histogram.
\begin{file}[SDH\_feat.hxx]
  \begin{lstlisting}[language=C]
      double
       *sumHist,
       *diffHist,
        Hist[2][511]
      ;
  \end{lstlisting}
\end{file}


\subsection{Public methods}
The class \textbf{SDH} has public methods for access, ($d$, $\theta$) definition
and some other utilities most for the internal use of the class, but still public.
Most of them are overloaded, but the most important ones are the following:

\begin{file}[SDH\_feat.hxx]
  \begin{lstlisting}[language=C]
      SDH(const cv::Mat& src);
      void set(int d, uint angle);
      double  set(const int which_one, int index, double value);
      double* at(const int which_one, int index);
      double* atRel(const int which_one, size_t index);
      void toCSV(std::string filename, unsigned int HEADER=0, bool append=false, std::string name="SDH");
      static int getSDH(const cv::Mat& src, SDH& sdh);
      static void computeFeatures(SDH& sdh);
      static void toCSV(std::vector<SDH>& sdh, std::string filename, unsigned int HEADER=SDH::ALL);
      static void toCSV_WriteHeader(std::string filename, unsigned int HEADER=SDH::ALL);
      void printFeatures(unsigned int HEADER=SDH::ALL);
  \end{lstlisting}
\end{file}

\subsubsection{The constructor}
There are other constructors, but here, the used in this experiment
initializes the class with a given image.
By default, the ($d$, $\theta$) pair is set to (1,0).
 
\subsubsection{The set method}
The \textbf{set} method is intended to change the value of a given
bin of any histogram.

\subsubsection{The at method}
The \textbf{at} method is intended to return a pointer to the element
at the $index$ position of any histogram. Index ranges are from $0$ to $510$
for \texttt{SDH::SUM}, and from $-255$ to $255$ for \texttt{SDH::DIFF}.

\subsubsection{The atRel method}
The \textbf{atRel} method is intended to return a pointer to the element
at the $index$ position of any histogram, but relative to the $0$
index used in C\/C\+\+. In other words, both \texttt{SDH::DIFF} and 
\texttt{SDH::SUM} start at 0.

\subsubsection{The getSDH method}
Here, the process of computing the SDH occurs, with a window-focused approach.
This function computes two submatrices which will be Op1 and Op2,
and then computes it's sum and difference matrices.

This method is overloaded allowing it to be called from the object with no
arguments or with the facility to set the ($d$, $\theta$) pair before
processing the image.

With logOn=true, it exports to csv the SDH histograms, normalized and
unnormalized.

\subsubsection{The computeFeatures method}
This method is where all the features are computed from the SDH histograms,
there's an overload for it to be called from the object with no arguments.

The most important, is that it's needed to be called to compute the features
after the SDH has been computed.

It's separate from the \texttt{getSDH} method because it's better having
step by step each process.

\subsubsection{The printFeatures method}
The \textbf{printFeatures} method is intended to print the features
of a \texttt{SDH} object, can be used with the macros described in 
\ref{control_variables}

\subsubsection{The toCSV and toCSV\_WriteHeader methods}
The \textbf{toCSV} method is intended to write the features of a
\texttt{SDH} object to a CSV file. Can be used with the macros described in
\ref{control_variables}

toCSV\_WriteHeaderC method is intended to write the header of a CSV
file with the features of a \texttt{SDH} object, then the user may just
append the features to the file if there are many \texttt{SDH} objects
or they're within a loop.

\subsection{Private methods}
\begin{file}[SDH\_feat.hxx]
  \begin{lstlisting}[language=C]
      void incrementHist(int which_one, int index, double step=1);
      void decrementHist(int which_one, int index, double step=1);
  \end{lstlisting}
\end{file}

\subsubsection{The incrementHist and decrementHist methods}
The \textbf{incrementHist} and \textbf{decrementHist} methods are
intended to increment or decrement a bin of any histogram. The user can specify
the step (by default, 1). The indexes here are the same as the \texttt{at}
method, from $0$ to $510$ for \texttt{SDH::SUM}, and from $-255$ to $255$
for \texttt{SDH::DIFF}.
