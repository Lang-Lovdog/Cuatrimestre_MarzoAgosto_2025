%!TeX root = MarquezSalazarBrandon-Tarea04-Reporte.tex
\section{Final thoughts}

Here the results for \textit{Solid.png}, show an interesting behaviour
compared to the \textit{Sunset-cirro.jpeg} and \textit{Natur-police.jpeg}.

As can be seen into every table, for \textit{Solid.png} almost every value
is the same no matter which paoir of ($d$, $\theta$) is used.

Something similar happens with the \textbf{mean} of each image, which, no mather
the pair of ($d$, $\theta$) is used, the mean is almost the same per image.

Most of the metrics, slightly change depending on the pair of ($d$, $\theta$) used.

On the other hand for the image \textit{Bus.jpeg}, with $d=1$, the \textbf{contrast} changes
abruptly. This exaggerate changes occur also with \textit{Natur-police.jpeg}
specially when looking at the pair of ($d$, $\theta$) = (2, 135°).

It seems redundant (in this experiment), having different pairs of ($d$, $\theta$)
for each image, except for the \textbf{contrast}.


Now, about how the numbers can be felt looking at the images, it's fair to
point that \textit{Solid.png} and \textit{Sunset-cirro.jpeg} have the higher
\textbf{mean} and \textbf{homogeneity}, and the smallest \textbf{contrast};
numbers which can be interpreted as less variation in the texture, which is
what can be seen.


Finally, it's good to point the oposite case, which indicates different
behaviour from \textit{Natur-police.jpeg} and \textit{Bus.jpeg} that
have more variations in the texture.

\subsection{Some future experiments}

It would be interesting to make more experiments with other images with
different textures, structures and illuminations, applying image enhancement
techniques, variations on light, reducing the noise, etc.

Also, it would be interesting to see how classifiers can behave with those
images, how dimensionality reduction algorithms can confirm if
there are actual redundancies within the obtained features.
