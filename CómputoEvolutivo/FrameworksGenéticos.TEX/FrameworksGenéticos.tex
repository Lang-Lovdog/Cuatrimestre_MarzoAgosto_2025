%! TeX program = xelatex
%! TeX TS-program = xelatex
\documentclass[10pt,dvipsnames]{beamer}
\usepackage{fontspec}
\setmainfont{QTBookmann}
\setsansfont{QTFrizQuad}
%\setmonofont{VictorMono Nerd Font Mono}
\setmonofont{VictorMono NF}
\usepackage{polyglossia}
\usepackage{xcolor}
\usepackage{dirtytalk}
\usepackage{graphicx}
\setdefaultlanguage{spanish}
\usepackage{multirow}
\usepackage{listings}
\usepackage{ragged2e}
\usepackage[most]{tcolorbox}
\usepackage[backend=biber]{biblatex}
\usetheme{Copenhagen}
%%%%%%%%%%%%%%%%%%%%%%%%%%%%%%%%%%%%%%%%%%%%%%%%%%%%%%%%%%%%%%%%%%%%%%%%
%\definecolor{mypurple}{RGB}{044,040,028}                              %
%\setbeamercolor*{palette primary}{use=structure,fg=white,bg=mypurple} %
%\definecolor{mypurple}{RGB}{195,084,023}                              %
\definecolor{mypurple}{RGB}{104,020,108}
\setbeamercolor*{palette primary}{use=structure,fg=white,bg=mypurple}
\setbeamercolor{normal text}{fg=white, bg=black}
\setbeamercolor{tcolorbox text}{fg=white, bg=black}
\setbeamertemplate{navigation symbols}{}                              %
%\setbeamercolor{itemize item}{fg=PineGreen}
\newcommand{\colouredcircle}{%
  \tikz{\useasboundingbox (-0.2em,-0.32em) rectangle(0.2em,0.32em);
        \draw[ball color=PineGreen!7!Plum!90!Red,shading=ball,line width=0.03em] (0,0) circle(0.18em);}}
\newcommand{\colouredcircledis}{%
  \tikz{\useasboundingbox (-0.2em,-0.32em) rectangle(0.2em,0.32em);
        \draw[ball color=PineGreen!7!Plum!20!Black,shading=ball,line width=0.03em] (0,0) circle(0.18em);}}
\setbeamertemplate{itemize item}{\colouredcircle}
%\renewcommand{\labelitemi}{\colouredcircle}
\setbeamercolor*{bibliography entry title}{color=Cyan}
%%%%%%%%%%%%%%%%%%%%%%%%%%%%%%%%%%%%%%%%%%%%%%%%%%%%%%%%%%%%%%%%%%%%%%%%
\usepackage{tikzpagenodes}
\setbeamertemplate{background canvas}{%
  \begin{tikzpicture}[inner sep=0pt,remember picture,overlay]
    \node at (current page.center) {\includegraphics[height=\paperheight,width=\paperwidth]{fondo}};
  \end{tikzpicture}
}

\makeatletter
\newtcolorbox{blur}[1][]{%
  #1,
  enhanced,
  remember,
  frame hidden,
  interior hidden,
  fonttitle=\bfseries\centering, 
  fontupper=\rmfamily\selectfont,
  coltext=white,
  underlay={
    \begin{tcbclipframe}
      \begin{scope}[inner sep=0pt,remember picture,overlay]
        \fill[white] (current page.south west) rectangle (current page.north east);
        \node[opacity=1] at (current page.center) {\includegraphics[height=\paperheight, width=\paperwidth]{blured}};
      \end{scope}
    \end{tcbclipframe}
   }
}
\makeatother
%%%%%%%%%%%%%%%%%%%%%%%%%%%%%%%%%%%%%%%%%%%%%%%%%%%%%%%%%%%%%%%%%%%%%%%%
\newcommand{\separador}[1]{
  \vskip-4pt
  \begin{center}
    \rule{0.9\linewidth}{#1}
  \end{center}
}

\lstset{
  basicstyle=\ttfamily,
  showstringspaces=false,
  breaklines=true
}
\bibliography{bibliografia}
\nocite{*}

\author{%
Brandon Marquez Salazar
}
\title{%
  Frameworks que implementan algoritmos genéticos (python)
}
\begin{document}
\maketitle
\justifying

  \section*{Introducción}
  \begin{frame}
    \only<1>{
      \begin{blur}

        Los algoritmos genéticos pertenecen a la categoría de eurísticos,
        permiten encontrar soluciones óptimas para un problema.
        \separador{2pt}

        Consisten, principalmente, en una implementación simplificada del
        proceso evolutivo natural.
        \separador{1pt}
  
        Existen diferentes frameworks para la implementación de algoritmos
        genéticos, generalmente en Python y en Matlab.
  
      \end{blur}
    }
    \only<2,3>{
      \begin{blur}
    \onslide*<2,3> {
        Existen diferentes frameworks para la implementación de algoritmos
        genéticos, generalmente en Python y en Matlab.

        En este documento se discutirán dos de ellos.
    }
    \onslide*<2>{
      \begin{itemize}
        \item DEAP \textit{Distributed Evolutionary Algorithms in Python},
          y es un módulo que implementa algoritmos evolutivos distribuidos. DEAP
          permite ejecutar procesos paralelos. \cite{nash51}
        \item [\colouredcircledis] \textcolor{Gray}{GAFT}
      \end{itemize}
    }
    \onslide*<3>{
      \begin{itemize}
        \item [\colouredcircledis] \textcolor{Gray}{DEAP}
        \item GAFT \textit{Genetic Algorithm Framework in Python}
          permite la implementación de algoritmos genéticos y la definición
          de individuos de manera flexible
      \end{itemize}
    }
      \end{blur}
    }
  \end{frame}


  \begin{frame} \section*{DEAP}
    \begin{blur}[title={DEAP}]

      DEAP permite agilizar la búsqueda
      de soluciones óptimas a un problema. Tiene un conjunto de
      herramientas para la abstracción del individuo y métodos para
      selección y cruza. \cite{nash51}

      A nivel de codificación, se concibe como una implementación
      explícita (comparándose con un pseudocódigo) en pos de la
      legibilidad. Así mismo, dando importancia a la sencillez y limpieza
      del código. \cite{ffmmdeap}

    \end{blur}
  \end{frame}

  \begin{frame}[fragile]
    \begin{blur}[title={Una demostración con DEAP}]

      Función de evaluación:
\begin{center} \begin{lstlisting}[language=python]
    def eval_func(individual):
      return sum(individual)
\end{lstlisting} \end{center}

    \end{blur}
  \end{frame}

  \begin{frame} \section*{GAFT}
    \begin{blur}[title={GAFT}]

    Promete la capacidad de ajustarse a la creación de soluciones
    específicas para problemas puntuales. 

    \end{blur}
  \end{frame}

  \begin{frame}
    \begin{blur}
      \printbibliography
    \end{blur}
  \end{frame}

\end{document}
