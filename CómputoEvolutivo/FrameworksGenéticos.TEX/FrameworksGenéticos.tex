\documentclass[10pt]{article}
\usepackage{fontspec}
\usepackage{xcolor}
\setmainfont{QTBookmann}
\usepackage{polyglossia}
\usepackage{graphicx}
\usepackage{dirtytalk}
\setdefaultlanguage[variant=mexican]{spanish}
\usepackage{multirow}
\bibliography{bibliografia}
\nocite{*}

\author{%
Brandon Marquez Salazar
}
\title{%
  Frameworks que implementan algoritmos genéticos (python)
}
\begin{document}
\maketitle

  \section*{Introduccióno}

  Los algoritmos genéticos son algoritmos heurísticos que permiten la
  optimización de problemas, a fin de encontrar una solución óptima ante
  la situación planteada. Consisten, principalmente, en una implementación
  simplificada del proceso evolutivo natural. Existen diferentes
  frameworks para la implementación de algoritmos genéticos, generalmente
  en Python y en Matlab.

  \section*{DEAP}

  DEAP significa \textit{Distributed Evolutionary Algorithms in Python},
  y es un módulo que implementa algoritmos evolutivos distribuidos. DEAP
  permite ejecutar procesos paralelos, lo que permite agilizar la búsqueda
  de soluciones óptimas a un problema. Tiene un conjunto de herramientas
  para la abstracción del individuo y métodos para selección y cruza.

  \section*{GAFT}

  \textit{Genetic Algorithm Framework in Python}, es un framework que
  permite la implementación de algoritmos genéticos de forma flexible,
  procurando ajustarse a las necesidades del usuario, permitiendo al
  usuario definir a los individuos de manera personalizable. Estas
  características prometen que el framework ofrezca soluciones específicas
  a problemas puntuales.

\end{document}
