%! TeX program = xelatex
%! TeX TS-program = xelatex
%! TeX root = ProgramaciónEvolutiva.tex
%%%%%%%%%%%%%%%%%%%%%%%%%%%%%%%%%%%%%%%%%%%%%%%%%%%%%%%%%%%%%%%%%%%%%%%%
% Define AntiqueWhite color
\definecolor{AntiqueWhite}{RGB}{250,235,215}
\definecolor{mypurple}{RGB}{104,020,108}
\setbeamercolor*{palette primary}{use=structure,fg=white,bg=mypurple}
\setbeamercolor{normal text}{fg=white, bg=black}
\setbeamercolor{tcolorbox text}{fg=white, bg=black}
\setbeamertemplate{navigation symbols}{}                              %
\newcommand{\colouredcircle}{%
  \tikz{\useasboundingbox (-0.2em,-0.32em) rectangle(0.2em,0.32em);
        \draw[ball color=PineGreen!7!Plum!90!Red,shading=ball,line width=0.03em] (0,0) circle(0.18em);}}
\newcommand{\colouredcircledis}{%
  \tikz{\useasboundingbox (-0.2em,-0.32em) rectangle(0.2em,0.32em);
        \draw[ball color=PineGreen!7!Plum!20!Black,shading=ball,line width=0.03em] (0,0) circle(0.18em);}}
\setbeamertemplate{itemize item}{\colouredcircle}
\setbeamercolor*{bibliography entry title}{fg=Yellow!80!White, bg=Black}
% Set beamer bibliography titles and numbers color to antique white
\setbeamercolor*{bibliography entry author}{fg=AntiqueWhite}
\setbeamercolor*{bibliography entry location}{fg=AntiqueWhite}
\setbeamercolor*{bibliography entry note}{fg=AntiqueWhite}
%% Numbering color to Pine Green
\setbeamercolor*{bibliography item}{fg=LimeGreen!80!White}
% Set beamer bibliography urls color to some magenta
\setbeamercolor*{bibliography entry url}{fg=Magenta}
% Captions color to rgba (0,255,0,1)
\definecolor{LimeGreen}{RGB}{0,255,0}
\setbeamercolor{caption name}{fg=LimeGreen}
%%%%%%%%%%%%%%%%%%%%%%%%%%%%%%%%%%%%%%%%%%%%%%%%%%%%%%%%%%%%%%%%%%%%%%%%
\usepackage{tikzpagenodes}
\setbeamertemplate{background canvas}{%
  \begin{tikzpicture}[inner sep=0pt,remember picture,overlay]
    \node at (current page.center) {\includegraphics[height=\paperheight,width=\paperwidth]{fondo}};
  \end{tikzpicture}
}%
\usetikzlibrary{arrows.meta, decorations.pathmorphing,positioning,trees}

\lstset{
    basicstyle=\ttfamily\small,
    breaklines=true,
    backgroundcolor=\color{black!98},
    keywordstyle=\color{Magenta},
    commentstyle=\color{AntiqueWhite},
    stringstyle=\color{Yellow!80!white},
    showstringspaces=false,
    frame=single,
    rulecolor=\color{mypurple},
    frameround=tttt,
    escapeinside={\%*}{*)}
}

\makeatletter
% Set victor mono as default ttfont
\newtcolorbox{blur}[1][]{%
  #1,
  enhanced,
  remember,
  breakable, % Already enabled (good!)
  frame hidden,
  interior hidden,
  fonttitle=\bfseries\centering, 
  fontupper=\rmfamily\selectfont,
  coltext=white,
  underlay={
    \begin{tcbclipframe}
      \begin{scope}[inner sep=0pt,remember picture,overlay]
        \fill[white] (frame.south west) rectangle (frame.north east); % Changed to `frame` (not `current page`)
        \node[opacity=1] at (frame.center) {\includegraphics[height=\paperheight, width=\paperwidth]{blured}};
      \end{scope}
    \end{tcbclipframe}
  }
}
\makeatother
%%%%%%%%%%%%%%%%%%%%%%%%%%%%%%%%%%%%%%%%%%%%%%%%%%%%%%%%%%%%%%%%%%%%%%%%
\newcommand{\separador}[1]{
  \vskip-4pt
  \begin{center}
    \rule{0.9\linewidth}{#1}
  \end{center}
}

\lstset{
  basicstyle=\ttfamily,
  showstringspaces=false,
  breaklines=true
}

\newcommand{\transparencia}[1]{%
  \begin{frame}
    \begin{blur}
      #1
    \end{blur}
  \end{frame}
}

% set bibliography background for text to be legible despite the background

\usepackage{etoolbox}
\newcommand{\setupblurbibliography}{%
  \pretocmd{\frametitle}{\blurbackground}{}{}%
  \setbeamertemplate{bibliography item}{\color{LimeGreen!80!White}\insertbiblabel}%
  \setbeamercolor{bibliography entry author}{fg=AntiqueWhite}%
  \setbeamercolor{bibliography entry title}{fg=Yellow!80!White}%
  \setbeamercolor{bibliography entry location}{fg=AntiqueWhite}%
  \setbeamercolor{bibliography entry note}{fg=AntiqueWhite}%
  \setbeamercolor{bibliography entry url}{fg=Magenta}%
}

\newtcblisting{beamerlst}[1][]{
    enhanced,
    breakable,
    listing only,
    listing options={
        style=beamerlisting,
        language=Python, % Default language
        #1
    },
    colback=black!85, % Matches your theme
    colframe=mypurple, % Your purple color
    fontupper=\ttfamily\small,
    arc=3mm, % Rounded corners
    boxrule=1pt,
    % Blur effect (optional):
    underlay={
        \begin{tcbclipframe}
        \fill[black!90] (frame.south west) rectangle (frame.north east);
        \node[opacity=0.6] at (frame.center) 
            {\includegraphics[width=\linewidth]{blured}};
        \end{tcbclipframe}
    }
}

% Supporting style definition
\lstdefinestyle{beamerlisting}{
    basicstyle=\ttfamily\footnotesize\color{white},
    keywordstyle=\color{Magenta},
    commentstyle=\color{AntiqueWhite},
    stringstyle=\color{Yellow!80!white},
    showstringspaces=false,
    breaklines=true,
    tabsize=2
}

\newcommand{\codebox}[2][]{%
    \begin{tcolorbox}[
        enhanced,
        colback=black!85,
        colframe=mypurple,
        arc=3mm,
        boxrule=1pt,
        #1
    ]
    \lstinputlisting{#2}
    \end{tcolorbox}%
}

% Smart blur background that works with frame breaks
\newcommand{\blurbackground}{%
  \begin{tikzpicture}[remember picture,overlay]
    % Calculate content area with margins
    \path ([xshift=1cm,yshift=-1cm]current page.north west) coordinate (top left);
    \path ([xshift=-1cm,yshift=1cm]current page.south east) coordinate (bottom right);
    
    % Frosted glass effect
    \fill[black!85,opacity=0.92] (top left) rectangle (bottom right);
    \node[opacity=0.8] at (current page.center) 
      {\includegraphics[width=\paperwidth-2cm,height=\paperheight-2cm]{blured}};
  \end{tikzpicture}%
}


\includeonly{01.-EvolutionaryComputing}

\bibliography{bibliografia}
\nocite{*}
