%! TeX program = xelatex
%! TeX TS-program = xelatex
\documentclass[10pt,dvipsnames]{beamer}
\usepackage{fontspec}
\setmainfont{QTBookmann}
\setsansfont{QTFrizQuad}
\setmonofont{VictorMono Nerd Font Mono}
\usepackage{polyglossia}
\usepackage[dvipsnames]{xcolor}
\usepackage{dirtytalk}
\usepackage{graphicx}
\setdefaultlanguage{spanish}
\usepackage{multirow}
\usepackage{listings}
\usepackage{ragged2e}
\usepackage[most]{tcolorbox}
\usepackage[]{biblatex}
\usetheme{Copenhagen}
\makeatletter
\DeclareRobustCommand{\cvdots}{%
  \vbox{
    \baselineskip3.7\p@\lineskiplimit\z@
    \kern-\p@
    \hbox{.}\hbox{.}\hbox{.}
  }}
\makeatother

\DeclareMathOperator{\sign}{sign}

%%% TikZ for neuron model
\tikzstyle{ninput}    =[circle,    draw, minimum size =01.7em,  color=BlueGreen,inner sep=0pt]
\tikzstyle{neuron}    =[circle,    draw, minimum size =01.7em,  color=BrickRed, inner sep=0pt]
\tikzstyle{actfun}    =[rectangle, draw, minimum size =01.7em,  color=Violet,   inner sep=0pt,
                                         minimum width=03.4em]  

\usetikzlibrary{
  arrows.meta,      % 箭头形状
  shapes.geometric, % 几何形状
  chains,           % 链式布局
  calc,             % 坐标计算
}
%%% FLOWCHART ELEMENTS
\usepackage{tikz-flowchart}
\tikzstyle{fc start}=[
  base,
  circle,
  minimum size=1em,
  fill=Black,
  inner sep=0
]
%%% FLOWCHART ELEMENTS

\tikzset{signfn/.pic={
  \draw (-1.0,-0.5) --
        ( 0.0,-0.5) --
        ( 0.0, 0.0) --
        ( 0.0, 0.5) --
        ( 1.0, 0.5);
}}


\author{%
Brandon Marquez Salazar
}
\title{%
  Evolutionary programming
}
\begin{document}
\maketitle
\justifying

  \section*{Introduction}
  \begin{frame}{Introduction}
    \begin{blur}

      \onslide<1,4>
      Evolutionary programming is part of evolutionary algorithms.

      \onslide<2,4>
      The concept of EAs comes from the study of what's called
      \textbf{complex adaptive systems}. 

      \onslide<3,4>
      Its first algorithmic implementation can be traced back to
      Lawrence and David Fogel's work.

    \end{blur}
  \end{frame}


  \begin{frame}{Basic evolutionary process}
  \begin{blur}
    \begin{itemize}
      \item One or more populations of \textbf{individuals competing} for
      limited resources.
      \item The notions of \textbf{dynamically changing} populations due
      to birth and death of individuals.
      \item A concept of \textbf{fitness} which reflects the ability of
      an individual to survive and reproduce.
      \item A concept of \textbf{variational inheritance}: offspring
      closely resemble their parents but are not identical.
    \end{itemize}
  \end{blur}
  \end{frame}

  \begin{frame}{}
  \begin{blur}
    \onslide<1,3>
    Evolutionary algorithms are useful for problem solving due to its
    \textit{complex adaptive behaviour} emulation. Not so different from
    how humans search for solutions to any problem:

    \onslide<2,3>
    \begin{itemize}
      \item Formulating solutions,
      \item then \textbf{evaluating} them:
      \item the best ones are used, the less useful ones are usually
      \textbf{discarded} and
      \item some of them are \textbf{\itshape mutated} or
      \textbf{\itshape recombined} to create new ones, maybe better.
    \end{itemize}
  \end{blur}
  \end{frame}

  \begin{frame}{Evolutionary programming}
  \begin{blur}
    Fogel proposed modeling individuals as finite state machines.

    Also, he proposed sexual and asexual reproduction of individuals
    in the algorithm.

    The program should start with a N number of initial individuals;
    and the next generation is determined by combining new individuals
    into a 2N population, raking them and selecting the best N
    individuals.
  \end{blur}
  \end{frame}

  \begin{frame}{Problem solving}
  \begin{blur}
    \onslide*<1> {
        Since simple rules could demonstrate complex behaviour and some
        sort of intelligent exploration in a space of possible solutions,
        we can use EAs to solve problems.
    }

    \onslide*<2> {
      Each problem should be modeled in order to find
      \begin{itemize}
        \item An individual definition that satifies the problem's
        solution space characterization.
        \item A fitness function that describes correctly our problem's
        behaviour.
        \item A \textbf{variation operator} that generates new
        individuals from existing ones.
        \item A \textbf{selection operator} that selects the best
        individuals from the population.
      \end{itemize}
    }
    \onslide*<3>{
      This can be applied, for example , not only to simple problems (like
      function maximization/minimization) but to more complex ones like
      ANN's weights tuning, scheduling, energy consumption optimization, etc.
    }
  \end{blur}
  \end{frame}

\end{document}
